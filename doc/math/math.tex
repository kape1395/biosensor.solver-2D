\documentclass[a4paper,10pt]{article}

\title{Equations}
\author{Karolis Petrauskas}
\usepackage{amsmath, amsthm, amssymb}

\begin{document}
\maketitle
\begin{abstract}
    Mathematical part of the solvers.
\end{abstract}

%%%%%%%%%%%%%%%%%%%%%%%%%%%%%%%%%%%%%%%%%%%%%%%%%%%%%%%%%%%%%%%%%%%%%%%%%%%%%%%%
%%%%%%%%%%%%%%%%%%%%%%%%%%%%%%%%%%%%%%%%%%%%%%%%%%%%%%%%%%%%%%%%%%%%%%%%%%%%%%%%
\section{Implicid 2D solver in the cartesian and cylindrical coordinate sstems}
\subsection{Mathematical model}
Lets define the following symbols.
$S$ is the substance concentration in time and two-dimensional space and
$R$ is a speed of the reaction.
Generic equation, that governs processes inside of area is:
\begin{equation}\label{eq.im2d.governing}
    \frac{\partial S}{\partial t} = D \Delta S + R.
\end{equation}
Here $\Delta$ is the Laplace operator. It has different forms in the different coordinate system.

%%%%%%%%%%%%%%%%%%%%%%%%%%%%%%%%%%%%%%%%%%%%%%%%%%%%%%%%%%%%%%%%%%%%%%%%%%%%%%%%
\subsubsection{Diffusion}

In the cartesian coordinate sytem $S = S(x, y, t)$ and $R = R(x, y, t)$.
\begin{equation}\label{eq.im2d.laplace.cartesian}
    \Delta S = \frac{\partial^2 S}{\partial x^2} + \frac{\partial^2 S}{\partial y^2}.
\end{equation}


In the cylindrical (r,z plane) coordinate system $S = S(r, z, t)$ and $R = R(r, z, t)$.
\begin{equation}\label{eq.im2d.laplace.cylindrical}
    \Delta S =
        \frac{1}{r} \frac{\partial}{\partial r}
        \left( r \frac{\partial f}{\partial r} \right) +
        \frac{\partial^2 f}{\partial z^2}.
\end{equation}



%%%%%%%%%%%%%%%%%%%%%%%%%%%%%%%%%%%%%%%%%%%%%%%%%%%%%%%%%%%%%%%%%%%%%%%%%%%%%%%%
\subsubsection{Reactions}

Michaelis-menten reaction:
\begin{equation}
    R_{mm} = <V_{max}, K_M, S, P>
\end{equation}
\begin{equation}
    R = \begin{cases}
            -\frac{V_{max} S}{K_M + S} & \text{in equation for substrate $S$,}
            \\
            +\frac{V_{max} S}{K_M + S} & \text{in equation for product $P$.}
        \end{cases}
\end{equation}
%
``Simple'' reaction:
\begin{equation}
    R_{s} = <k, S_{s1}, S_{s2}, S_{r1}, S_{r2}>
\end{equation}
\begin{equation}
    R =
        - \sum_{R_s : S \in \{S_{s1}, S_{s2}\}} k S_{s1} S_{s2}
        + \sum_{R_s : S \in \{S_{p1}, S_{p2}\}} k S_{p1} P_{p2}
\end{equation}



%%%%%%%%%%%%%%%%%%%%%%%%%%%%%%%%%%%%%%%%%%%%%%%%%%%%%%%%%%%%%%%%%%%%%%%%%%%%%%%%
\subsubsection{Bounds}
Constant condition:
\begin{equation}
    S(x, y, t) = C,\quad (x, y) \in \Gamma.
\end{equation}
Non-leakage (wall) condition:
\begin{equation}
    \left.\frac{\partial S}{\partial n}\right|_{\Gamma} = 0.
\end{equation}
Merge condition:
\begin{equation}
    D_A \left.\frac{\partial S_A}{\partial n}\right|_{\Gamma} =
    D_B \left.\frac{\partial S_B}{\partial n}\right|_{\Gamma}.
\end{equation}





%%%%%%%%%%%%%%%%%%%%%%%%%%%%%%%%%%%%%%%%%%%%%%%%%%%%%%%%%%%%%%%%%%%%%%%%%%%%%%%%
\subsection{Finite differences}

The partial derivate from \eqref{eq.im2d.governing} by time is aproximated as follows:
\begin{equation}\label{eq.im2d.time}
    \frac{\partial S}{\partial t}
    \approx
    \frac{S_{i,j,k} - S_{i,j,k-1}}{\tau}
\end{equation}


The Laplace operator, formulated in the cartesian coordinate system \eqref{eq.im2d.laplace.cartesian}
is aproximated as follows:
\begin{multline}\label{eq.im2d.diffusion.cartesian}
    \frac{\partial^2 S}{\partial x^2} + \frac{\partial^2 S}{\partial y^2}
    \approx
    \\%%%%%%%%%%%%%%%%%%%%%%%%%%%
    \approx
    \frac{S_{i+1,j,k} - 2 S_{i,j,k} + S_{i-1,j,k}}{g^2} +
    \frac{S_{i,j+1,k} - 2 S_{i,j,k} + S_{i,j-1,k}}{h^2}
    =
    \\%%%%%%%%%%%%%%%%%%%%%%%%%%%
    =
    -2\frac{g^2 + h^2}{g^2 h^2} S_{i,j,k}
    +\frac{1}{g^2}              S_{i+1,j,k}
    +\frac{1}{g^2}              S_{i-1,j,k}
    +\frac{1}{h^2}              S_{i,j+1,k}
    +\frac{1}{h^2}              S_{i,j-1,k}
\end{multline}

Cylindrical coordinate system,  $(r,z)$ plane. $S = S(r, z, t)$.
Case one -- non simetrical by inner $r$.
Note that $r_{i+1} = r_i + g$.
\begin{multline}
    \frac{1}{r} \frac{\partial}{\partial r}
    \left( r \frac{\partial S}{\partial r} \right) +
    \frac{\partial^2 S}{\partial z^2}
    \approx
    \\%%%%%%%%%%%%%%%%%%%%%%%%%%%
    \approx
    \frac{1}{r}\frac{\partial}{\partial r}\left(r_i \frac{S_{i,j,k} - S_{i-1,j,k}}{g} \right) +
    \frac{\partial^2 S}{\partial z^2}
    \approx
    \\%%%%%%%%%%%%%%%%%%%%%%%%%%%
    \approx
    \frac{ r_{i+1}\frac{S_{i+1,j,k} - S_{i,j,k}}{g} - r_{i}\frac{S_{i,j,k} - S_{i-1,j,k}}{g}}{r_i g} +
    \frac{\partial^2 S}{\partial z^2}
    \approx
    \\%%%%%%%%%%%%%%%%%%%%%%%%%%%
    \approx
    \frac{ r_{i+1}S_{i+1,j,k} - (r_{i+1} + r_i)S_{i,j,k} + r_i S_{i-1,j,k} }{r_i g^2} +
    \frac{S_{i,j+1,k} - 2 S_{i,j,k} + S_{i,j-1,k}}{h^2}
    =
    \\%%%%%%%%%%%%%%%%%%%%%%%%%%%
    =
    \frac{ (r_i + g)S_{i+1,j,k} - (2 r_i + g)S_{i,j,k} + r_i S_{i-1,j,k} }{r_i g^2} +
    \frac{S_{i,j+1,k} - 2 S_{i,j,k} + S_{i,j-1,k}}{h^2}
    =
    \\%%%%%%%%%%%%%%%%%%%%%%%%%%%
    =
    -\frac{2 (h^2 + g^2) r_i + g h^2}{r_i g^2 h^2}  S_{i,j,k}
    +\frac{r_i + g}{r_i g^2}                        S_{i+1,j,k}
    +\frac{1}{g^2}                                  S_{i-1,j,k}
    +\frac{1}{h^2}                                  S_{i,j+1,k}
    +\frac{1}{h^2}                                  S_{i,j-1,k}
\end{multline}
Case two -- symetrical by inner $r$.
Note that $r_{i+1/2} = r_i + \frac{g}{2}$ and $r_{i-1/2} = r_i - \frac{g}{2}$.
The difference from tme previous case is in the second equation,
here we replaced $r$ with $r_{i-1/2}$ instead of $r_i$.
\begin{multline}\label{eq.im2d.diffusion.cylindrical}
    \frac{1}{r} \frac{\partial}{\partial r}
    \left( r \frac{\partial S}{\partial r} \right) +
    \frac{\partial^2 S}{\partial z^2}
    \approx
    \\%%%%%%%%%%%%%%%%%%%%%%%%%%%
    \approx
    \frac{1}{r}\frac{\partial}{\partial r}\left(r_{i-1/2} \frac{S_{i,j,k} - S_{i-1,j,k}}{g} \right) +
    \frac{\partial^2 S}{\partial z^2}
    \approx
    \\%%%%%%%%%%%%%%%%%%%%%%%%%%%
    \approx
    \frac{ r_{i+1/2}\frac{S_{i+1,j,k} - S_{i,j,k}}{g} - r_{i-1/2}\frac{S_{i,j,k} - S_{i-1,j,k}}{g}}{r_i g} +
    \frac{\partial^2 S}{\partial z^2}
    \approx
    \\%%%%%%%%%%%%%%%%%%%%%%%%%%%
    \approx
    \frac{ r_{i+1/2}S_{i+1,j,k} - (r_{i+1/2} + r_{i-1/2})S_{i,j,k} + r_{i-1/2} S_{i-1,j,k} }{r_i g^2} +
    \frac{S_{i,j+1,k} - 2 S_{i,j,k} + S_{i,j-1,k}}{h^2}
    =
    \\%%%%%%%%%%%%%%%%%%%%%%%%%%%
    =
    \frac{ (r_i + \frac{g}{2})S_{i+1,j,k} - 2 r_i S_{i,j,k} + (r_i - \frac{g}{2}) S_{i-1,j,k} }{r_i g^2} +
    \frac{S_{i,j+1,k} - 2 S_{i,j,k} + S_{i,j-1,k}}{h^2}
    =
    \\%%%%%%%%%%%%%%%%%%%%%%%%%%%
    =
    -2\frac{g^2 + h^2}{g^2 h^2}         S_{i,j,k}
    +\frac{r_i + \frac{g}{2}}{r_i g^2}  S_{i+1,j,k}
    +\frac{r_i - \frac{g}{2}}{r_i g^2}  S_{i-1,j,k}
    +\frac{1}{h^2}                      S_{i,j+1,k}
    +\frac{1}{h^2}                      S_{i,j-1,k}
\end{multline}



%%%%%%%%%%%%%%%%%%%%%%%%%%%%%%%%%%%%%%%%%%%%%%%%%%%%%%%%%%%%%%%%%%%%%%%%%%%%%%%%
\subsubsection{Alternating directions and tridiagonal matrixes}
The main equation system for one area is:
\begin{equation}
    a S + b S + c S = f
\end{equation}
here:
\begin{equation}
    a = a_D,\quad
    b = b_T + b_D,\quad
    c = c_D,\quad
    f = f_T + f_D + f_R.
\end{equation}
Functions $a_D, b_T, b_D, c_D, f_T, f_D$ and $ f_R$ are defined bellow.
Coefficients for $\Delta$ are taken from \eqref{eq.im2d.diffusion.cartesian}
and \eqref{eq.im2d.diffusion.cylindrical}.\\
From \eqref{eq.im2d.time} we get:
\begin{equation}
    b_T = -\frac{2}{\tau},\qquad
    f_T = \begin{cases}
            -\frac{2 S_{i,j,k-1  }}{\tau} & \text{for first direction};\\
            -\frac{2 S_{i,j,k-0.5}}{\tau} & \text{for second direction.}
          \end{cases}
\end{equation}
In the cylindrical coordinate system, by the coordinate $r$ (to find $S_{i,j,k-0.5}$):
\begin{multline}
    a_D = D \frac{r_i - \frac{g}{2}}{r_i g^2},\quad
    b_D = -\frac{2 D}{g^2},\quad
    c_D = D \frac{r_i + \frac{g}{2}}{r_i g^2},\\
    f_D = -D \frac{S_{i,j+1,k-1} - 2 S_{i,j,k-1} + S_{i,j-1,k-1}}{h^2}.
\end{multline}
In the cylindrical coordinate system, by the coordinate $z$ (to find $S_{i,j,k}$):
\begin{multline}
    a_D = \frac{D}{h^2},\quad
    b_D = -\frac{2 D}{h^2},\quad
    c_D = \frac{D}{h^2},\\
    f_D = - D \frac{ (r_i + \frac{g}{2})S_{i+1,j,k-0.5} - 2 r_i S_{i,j,k-0.5} + (r_i - \frac{g}{2}) S_{i-1,j,k-0.5} }{r_i g^2}.
\end{multline}
In the cartesian coordinate system, by coordinate $x$ (to find $S_{i,j,k-0.5}$):
\begin{multline}
    a_D = \frac{D}{g^2},\quad
    b_D = -\frac{2 D}{g^2},\quad
    c_D = \frac{D}{g^2},\\
    f_D = - D \frac{S_{i,j+1,k-1} - 2 S_{i,j,k-1} + S_{i,j-1,k-1}}{h^2}.
\end{multline}
In the cartesian coordinate system, by coordinate $y$ (to find $S_{i,j,k}$):
\begin{multline}
    a_D = \frac{D}{h^2},\quad
    b_D = -\frac{2 D}{h^2},\quad
    c_D = \frac{D}{h^2},\\
    f_D = - D \frac{S_{i+1,j,k-0.5} - 2 S_{i,j,k-0.5} + S_{i-1,j,k-0.5}}{g^2}.
\end{multline}
For Michaelis-Menten reaction (for both directions is the same):
\begin{equation}
    f_R =
        \begin{cases}
            +\frac{V_{max} S_{i,j,k-1}}{K_M + S_{i,j,k-1}} &\text{for a substrate};\\
            -\frac{V_{max} S_{i,j,k-1}}{K_M + S_{i,j,k-1}} &\text{for a product}.
        \end{cases}
\end{equation}
For ``Simple'' reaction:
\begin{equation}
    aaa
\end{equation}







%%%%%%%%%%%%%%%%%%%%%%%%%%%%%%%%%%%%%%%%%%%%%%%%%%%%%%%%%%%%%%%%%%%%%%%%%%%%%%%%
%%%%%%%%%%%%%%%%%%%%%%%%%%%%%%%%%%%%%%%%%%%%%%%%%%%%%%%%%%%%%%%%%%%%%%%%%%%%%%%%
\end{document}
