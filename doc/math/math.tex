\documentclass[a4paper,10pt]{article}

\title{Equations}
\author{Karolis Petrauskas}
\usepackage{amsmath, amsthm, amssymb}

\begin{document}
\maketitle
\begin{abstract}
    Mathematical part of the solvers.
\end{abstract}

%%%%%%%%%%%%%%%%%%%%%%%%%%%%%%%%%%%%%%%%%%%%%%%%%%%%%%%%%%%%%%%%%%%%%%%%%%%%%%%%
%%%%%%%%%%%%%%%%%%%%%%%%%%%%%%%%%%%%%%%%%%%%%%%%%%%%%%%%%%%%%%%%%%%%%%%%%%%%%%%%
\section{Implicid 2D solver in the cartesian coordinates}
Lets define the following symbols.
$S$ is the substance concentration in time and two-dimensional space and
$R$ is a speed of the reaction.
Generic equation, that governs processes inside of area is:
\begin{equation}
    \frac{\partial S}{\partial t} = D \Delta S + R.
\end{equation}
Here $\Delta$ is the Laplace operator. It has different forms in the different coordinate system.

%%%%%%%%%%%%%%%%%%%%%%%%%%%%%%%%%%%%%%%%%%%%%%%%%%%%%%%%%%%%%%%%%%%%%%%%%%%%%%%%
\subsection{Diffusion}

In the cartesian coordinate sytem $S = S(x, y, t)$ and $R = R(x, y, t)$.
\begin{equation}
    \Delta S = \frac{\partial^2 S}{\partial x^2} + \frac{\partial^2 S}{\partial y^2}.
\end{equation}


In the cylindrical (r,z plane) coordinate system $S = S(r, z, t)$ and $R = R(r, z, t)$.
\begin{equation}
    \Delta S =
        \frac{1}{r} \frac{\partial}{\partial r}
        \left( r \frac{\partial f}{\partial r} \right) +
        \frac{\partial^2 f}{\partial z^2}.
\end{equation}

%%%%%%%%%%%%%%%%%%%%%%%%%%%%%%%%%%%%%%%%%%%%%%%%%%%%%%%%%%%%%%%%%%%%%%%%%%%%%%%%
\subsection{Reactions}

Michaelis-menten reaction:
\begin{equation}
    R = \begin{cases}
            -\frac{V_{max} S}{K_M + S} & \text{in equation for substrate $S$,}
            \\
            +\frac{V_{max} S}{K_M + S} & \text{in equation for product $P$.}
        \end{cases}
\end{equation}
%
``Simple'' reaction:
\begin{equation}
    R = \sum_i k_{R_i} S_{R_{i,a}} S_{R_{i,b}} - \sum_j k_{R_j} P_{R_{j,a}} P_{R_{j,b}}
\end{equation}



%%%%%%%%%%%%%%%%%%%%%%%%%%%%%%%%%%%%%%%%%%%%%%%%%%%%%%%%%%%%%%%%%%%%%%%%%%%%%%%%
\subsection{Finite differences}

\begin{equation}
    \frac{\partial S}{\partial t}
    \approx
    \frac{S_{i,j,k} - S_{i,j,k-1}}{\tau}
\end{equation}
\begin{multline}
    \frac{\partial^2 S}{\partial x^2} + \frac{\partial^2 S}{\partial y^2}
    \approx
    \\%%%%%%%%%%%%%%%%%%%%%%%%%%%
    \approx
    \frac{S_{i+1,j,k} - 2 S_{i,j,k} + S_{i-1,j,k}}{g^2} +
    \frac{S_{i,j+1,k} - 2 S_{i,j,k} + S_{i,j-1,k}}{h^2}
    =
    \\%%%%%%%%%%%%%%%%%%%%%%%%%%%
    =
    -2\frac{g^2 + h^2}{g^2 h^2} S_{i,j,k}
    +\frac{1}{g^2}              S_{i+1,j,k}
    +\frac{1}{g^2}              S_{i-1,j,k}
    +\frac{1}{h^2}              S_{i,j+1,k}
    +\frac{1}{h^2}              S_{i,j-1,k}
\end{multline}

Cylindrical coordinate system,  $(r,z)$ plane. $S = S(r, z, t)$.
Case one -- non simetrical by inner $r$.
Note that $r_{i+1} = r_i + g$.
\begin{multline}
    \frac{1}{r} \frac{\partial}{\partial r}
    \left( r \frac{\partial S}{\partial r} \right) +
    \frac{\partial^2 S}{\partial z^2}
    \approx
    \\%%%%%%%%%%%%%%%%%%%%%%%%%%%
    \approx
    \frac{1}{r}\frac{\partial}{\partial r}\left(r_i \frac{S_{i,j,k} - S_{i-1,j,k}}{g} \right) +
    \frac{\partial^2 S}{\partial z^2}
    \approx
    \\%%%%%%%%%%%%%%%%%%%%%%%%%%%
    \approx
    \frac{ r_{i+1}\frac{S_{i+1,j,k} - S_{i,j,k}}{g} - r_{i}\frac{S_{i,j,k} - S_{i-1,j,k}}{g}}{r_i g} +
    \frac{\partial^2 S}{\partial z^2}
    \approx
    \\%%%%%%%%%%%%%%%%%%%%%%%%%%%
    \approx
    \frac{ r_{i+1}S_{i+1,j,k} - (r_{i+1} + r_i)S_{i,j,k} + r_i S_{i-1,j,k} }{r_i g^2} +
    \frac{S_{i,j+1,k} - 2 S_{i,j,k} + S_{i,j-1,k}}{h^2}
    =
    \\%%%%%%%%%%%%%%%%%%%%%%%%%%%
    =
    \frac{ (r_i + g)S_{i+1,j,k} - (2 r_i + g)S_{i,j,k} + r_i S_{i-1,j,k} }{r_i g^2} +
    \frac{S_{i,j+1,k} - 2 S_{i,j,k} + S_{i,j-1,k}}{h^2}
    =
    \\%%%%%%%%%%%%%%%%%%%%%%%%%%%
    =
    -\frac{2 (h^2 + g^2) r_i + g h^2}{r_i g^2 h^2}  S_{i,j,k}
    +\frac{r_i + g}{r_i g^2}                        S_{i+1,j,k}
    +\frac{1}{g^2}                                  S_{i-1,j,k}
    +\frac{1}{h^2}                                  S_{i,j+1,k}
    +\frac{1}{h^2}                                  S_{i,j-1,k}
\end{multline}
Case two -- symetrical by inner $r$.
Note that $r_{i+1/2} = r_i + \frac{g}{2}$ and $r_{i-1/2} = r_i - \frac{g}{2}$.
The difference from tme previous case is in the second equation,
here we replaced $r$ with $r_{i-1/2}$ instead of $r_i$.
\begin{multline}
    \frac{1}{r} \frac{\partial}{\partial r}
    \left( r \frac{\partial S}{\partial r} \right) +
    \frac{\partial^2 S}{\partial z^2}
    \approx
    \\%%%%%%%%%%%%%%%%%%%%%%%%%%%
    \approx
    \frac{1}{r}\frac{\partial}{\partial r}\left(r_{i-1/2} \frac{S_{i,j,k} - S_{i-1,j,k}}{g} \right) +
    \frac{\partial^2 S}{\partial z^2}
    \approx
    \\%%%%%%%%%%%%%%%%%%%%%%%%%%%
    \approx
    \frac{ r_{i+1/2}\frac{S_{i+1,j,k} - S_{i,j,k}}{g} - r_{i-1/2}\frac{S_{i,j,k} - S_{i-1,j,k}}{g}}{r_i g} +
    \frac{\partial^2 S}{\partial z^2}
    \approx
    \\%%%%%%%%%%%%%%%%%%%%%%%%%%%
    \approx
    \frac{ r_{i+1/2}S_{i+1,j,k} - (r_{i+1/2} + r_{i-1/2})S_{i,j,k} + r_{i-1/2} S_{i-1,j,k} }{r_i g^2} +
    \frac{S_{i,j+1,k} - 2 S_{i,j,k} + S_{i,j-1,k}}{h^2}
    =
    \\%%%%%%%%%%%%%%%%%%%%%%%%%%%
    =
    \frac{ (r_i + \frac{g}{2})S_{i+1,j,k} - 2 r_i S_{i,j,k} + (r_i - \frac{g}{2}) S_{i-1,j,k} }{r_i g^2} +
    \frac{S_{i,j+1,k} - 2 S_{i,j,k} + S_{i,j-1,k}}{h^2}
    =
    \\%%%%%%%%%%%%%%%%%%%%%%%%%%%
    =
    -2\frac{g^2 + h^2}{g^2 h^2}         S_{i,j,k}
    +\frac{r_i + \frac{g}{2}}{r_i g^2}  S_{i+1,j,k}
    +\frac{r_i - \frac{g}{2}}{r_i g^2}  S_{i-1,j,k}
    +\frac{1}{h^2}                      S_{i,j+1,k}
    +\frac{1}{h^2}                      S_{i,j-1,k}
\end{multline}
So:
\begin{table}
\begin{tabular}{c|cc|cc|}
    Var & $cyl^1$ & $cyl^2$ & $dec^1$ & $dec^2$
    \\\hline
    $a_D$
        & $\frac{r_i - \frac{g}{2}}{r_i g^2}$
        & $\frac{1}{h^2}$
    \\\hline
    $b_D$
        & $-\frac{2}{g^2}$
        & $-\frac{2}{h^2}$
    \\\hline
    $c_D$
        & $\frac{r_i + \frac{g}{2}}{r_i g^2}$
        & $\frac{1}{h^2}$
    \\\hline
\end{tabular}
\end{table}

\begin{align}
    a_D &= \frac{r_i - \frac{g}{2}}{r_i g^2} \\
    b_D &= \frac{2}{g^2} \\
    c_D &= \frac{r_i + \frac{g}{2}}{r_i g^2}
\end{align}




%%%%%%%%%%%%%%%%%%%%%%%%%%%%%%%%%%%%%%%%%%%%%%%%%%%%%%%%%%%%%%%%%%%%%%%%%%%%%%%%
%%%%%%%%%%%%%%%%%%%%%%%%%%%%%%%%%%%%%%%%%%%%%%%%%%%%%%%%%%%%%%%%%%%%%%%%%%%%%%%%
\end{document}
